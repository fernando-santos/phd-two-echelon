\documentclass[12pt]{article}

\usepackage{amsmath}
\usepackage{amsfonts}
\usepackage[brazil]{babel}
\usepackage[latin1]{inputenc}
\usepackage[T1]{fontenc}
\usepackage{geometry}
\geometry{letterpaper,left=2.5cm,right=2.5cm,top=2.0cm,bottom=1.0cm}
\usepackage{graphicx}
\sloppy

\begin{document}

\noindent\textbf{Dados do problema:}
\begin{table}[!htb]
\begin{tabular}{cl}
\vspace{1mm}
$V = \{0, V_s, V_c\}$ & $V_s$ e $V_c$ s\~ao conjuntos dos v\'ertices satellities e customers, respectivamente \\
$K_1$ & Conjunto de ve\'iculos dispon\'ivel para o $1^o$ n\'ivel de roteamento \\
$K_2$ & Conjunto de ve\'iculos dispon\'ivel para o $2^o$ n\'ivel de roteamento \\
$m_s$ & Quantidade m\'axima de ve\'iculos que podem partir do satellite $s$ \\
$Q_1$ & Capacidade homog\^enea de todos os ve\'iculos do $1^o$ n\'ivel de roteamento \\
$Q_2$ & Capacidade homog\^enea de todos os ve\'iculos do $2^o$ n\'ivel de roteamento \\
$d_i$ & demanda do customer $i$ \\
$a_{ir}$ & Par\^ametro que indica se a rota $r$ passa (assume valor $1$) no v\'ertice $i$ ou n\~ao \\
$T_s$ & Custo unit\'ario para carregar/descarregar mercadorias no satellite $s$
\end{tabular}
\end{table}

\noindent\textbf{Vari\'aveis de decis\~ao:}
\begin{table}[!htb]
\begin{tabular}{rl}
\vspace{1mm}
$\lambda_{r_1} = $ & N\'umero de ve\'iculos que executam a rota $r$ de n\'ivel 1\\
\vspace{2mm}
$\delta^{sr_1} = $ & Quantidade de mercadorias recebida pelo satellite $s$ pela rota $r$ de nivel 1 \\
\vspace{2mm}
$\gamma^s_{r_2} = $ &
$\left\{
\begin{array} {l}
1 - \textrm{ A rota } r \textrm{ de nivel 2 que parte do satellite } s \textrm{ para atender os consumidores \'e usada } \\
0 - \textrm{ Caso contr\'ario}
\end{array} \right.
$
\end{tabular}
\end{table}

\noindent\textbf{Modelo:}
\vspace{5mm}
$$\min \,\, \sum\limits_{r_1 \in R_1} {c_{r_1} \lambda_{r_1}} + \sum\limits_{s \in V_s} {\sum\limits_{r_2 \in R_2}{c_{r_2} \gamma^s_{r_2}}} + \sum\limits_{s \in V_s} { \sum\limits_{r_1 \in R_1} T_s \delta^{sr_1} }$$
\begin{eqnarray}
\label{primal01}
  \sum\limits_{ r_1 \in R_1 } { \lambda_{r_1} } \le |K_1| \\
\label{primal02}
  \sum\limits_{ s \in V_s } { \sum\limits_{ r_2 \in R_2 } { \gamma^s_{r_2} } } \le |K_2| \\
\label{primal03}
  \sum\limits_{ r_2 \in R_2 } { \gamma^s_{r_2} } \le m_s & \hspace{1cm} & \forall s \in V_s \\
\label{primal04}
  \sum\limits_{ s \in V_s } { \sum\limits_{ r_2 \in R_2 } { a_{ir_2} \gamma^s_{r_2} } } = 1 & \hspace{1cm} & \forall i \in V_c \\
\label{primal05}
  \sum\limits_{ r_2 \in R_2 } { \gamma^s_{r_2} \sum\limits_{ i \in V_c }{ a_{ir_2} d_i } } = \sum\limits_{ r_1 \in R_1 }{\delta^{sr_1}} & \hspace{1cm} & \forall s \in V_s \\
\label{primal06}
  \sum\limits_{ s \in S } { a_{sr_1} \delta^{sr_1} \leq Q_1 \lambda_{r_1} } & \hspace{1cm} & \forall r_1 \in R_1 \\
\label{primal07}
  \sum\limits_{ r_1 \in R_1 } { \sum\limits_{ s \in V_s } { (1-a_{sr_1})\delta^{sr_1} } } = 0 \\
\label{primal08}
  \lambda \in \mathbb{B}^{|R_1|}, \gamma \in \mathbb{B}^{|V_s||R_2|}, \delta \in \mathbb{N}^{|V_s||R_1|}
\end{eqnarray}

\end{document}
